%Two resources useful for abstract writing.
% Guidance of how to write an abstract/summary provided by Nature: https://cbs.umn.edu/sites/cbs.umn.edu/files/public/downloads/Annotated_Nature_abstract.pdf %https://writingcenter.gmu.edu/guides/writing-an-abstract
\chapter*{\center \Large  Abstract}
%%%%%%%%%%%%%%%%%%%%%%%%%%%%%%%%%%%%%%
% Replace all text with your text
%%%%%%%%%%%%%%%%%%%%%%%%%%%%%%%%%%%

With the rising incidence of infectious illness, there is a necessity for a strong 
predictive analysis model. Diseases related to the lungs are increasing with the increase in air 
pollution. This research paper focuses on four types of lung disease 
that is bacterial pneumonia, corona virus disease, tuberculosis, viral 
pneumonia, These are all related to the lungs and are identified using different lungs X-ray 
images. To predict the type of 
disease related to the lungs a deep learning ANN is trained to classify lung image data.  We train a standard Convolutional Neural Network on the data, which helps extract relevant information from the given input image during 
training and prediction. The research challenges that are recognized help in understanding the 
need for computation resources in the healthcare sector and the potential of predictive 
analytics and modeling in understanding current healthcare management and avoiding future 
outbreaks. The evaluation of the model is conducted 
using different classification metrics such as f1 score and accuracy score, For binary classifiers, the Receiver Operating Characteristic ROC curve and its corresponding Area Under the Curve AUC-ROC provide information about how well a model can discriminate. Confusion matrices also provide a thorough analysis of forecasts,  which help to 
understand how accurately the model can identified.

