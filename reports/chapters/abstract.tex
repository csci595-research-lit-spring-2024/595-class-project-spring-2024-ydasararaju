%Two resources useful for abstract writing.
% Guidance of how to write an abstract/summary provided by Nature: https://cbs.umn.edu/sites/cbs.umn.edu/files/public/downloads/Annotated_Nature_abstract.pdf %https://writingcenter.gmu.edu/guides/writing-an-abstract
\chapter*{\center \Large  Abstract}
%%%%%%%%%%%%%%%%%%%%%%%%%%%%%%%%%%%%%%
% Replace all text with your text
%%%%%%%%%%%%%%%%%%%%%%%%%%%%%%%%%%%

With the rising incidence of infectious illness, there is a strong necessity for a strong predictive analysis model that works more efficiently and accurately as compared to manual analysis. Difference diseases related to the lungs are rising with the increase in air pollution and bad health which need to be analyzed as quickly as possible and precautions need to be taken to avoid it from growing. This research paper focuses on four types of lung disease that is “Bacterial Pneumonia”, “Corona Virus Disease”, “Normal”, “Tuberculosis”, “Viral Pneumonia” these are all related to the lungs and are identified using different lungs X-ray images where the analysis of the type of disease can be performed using different computation algorithm which will help in better analyzing of the disease as the computer can be deployed in various health care sector and is trained using much larger dataset. To predict the type of disease related to the lungs a deep learning model is trained which is based upon Convolutional Neural Network which helps extract relevant information from the given input image during training and prediction. The research challenges that are recognized help in understanding the need for computation resources in the healthcare sector and the potentiation of predictive analytics and modeling in understanding current healthcare management and avoiding future outbreaks of the disease across the globe and the sooner relevant precautions are taken the sooner the outbreak of the disease can be avoided. The evaluation of the model is conducted using different classification metrics such as f1 score and accuracy score which help to understand how accurately the model can identify the given disease. 

