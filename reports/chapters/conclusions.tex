\chapter{Conclusions and Future Work}
\label{ch:con}
\section{Conclusions}
For this research, three different Recurrent neural network models are proposed which were utilized to train on the lung data-related outbreak dataset obtained from an online source Kaggle. These models were trained on various aspects and it was seen that the MobileNet model gave the best prediction accuracy out of the other models which are VGG 19 and ResNet chosen for this research. Building such advanced models will revolutionize the working of healthcare institutions in the future as they can detect these deadly diseases at their early stage and thus prevent them with effective measures. Consequently, the standard of care for the impacted patient may be enhanced. 

\section{Future work}
It is crucial to investigate several approaches to increase the prediction model's accuracy and reliability. First, adding more diverse data from different eras and locations to the databases can provide a more comprehensive understanding of the historical and regional evolution of lung disease epidemics. More accurate predictions could result from improving data augmentation methods to more accurately capture the variety and intricacy of lung pictures. It is imperative to evaluate the model in actual healthcare environments and verify its success in early diagnosis and treatment of lung disorders. Experimenting with deep learning architectures other than VGG19, MobileNet, and ResNet may yield fresh perspectives and improve prediction accuracy. By combining many models, experimenting with ensemble learning techniques may also enhance predictions by using the advantages of each approach.  The field of lung disease prediction has the potential to greatly improve patient treatment and international public health outcomes via continuous innovation and improvement of predictive analytics tools. It may be possible to find new approaches to improving prediction models and increasing their effectiveness in managing complex healthcare data by consistently investigating different optimization algorithms.