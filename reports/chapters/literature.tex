\chapter{Literature Review}
\label{ch:lit_rev} %Label of the chapter lit rev. The key ``ch:lit_rev'' can be used with command \ref{ch:lit_rev} to refer this Chapter.

In recent years, the growth in different types of infectious diseases has led to the urgent need for any analysis method to understand and cure the disease where the most effective analysis can be performed using predictive models as manual analysis has many drawbacks such as its time-consuming and is hard to analyze the complexity of the illness when it comes to disease cause directly into the respiratory system \cite{haux2019research}. The major agent for illness in the respiratory system is the pollution present in the air and the unhygienic that lead and since some illnesses can be even passed using genetic or body touch therefore the effect count of such diseases grows at a rate of exponential \cite{madubueze2020controlling}. In the past decade, the traditional method has been used to understand the root cause of any given respiratory disease, and based on analysis proper cure is found but with the evolving nature of the illness over time the limitation can be seen where the analysis can take a longer time and same goes for cure in which the best cure regarding the illness need to figure out along with which one important factor to be noted is the predictive model should avoid False Negative for example if there is a minor chance that the user has cancer the model should predict that there is cancer along with probability as it would help in taking proper medication to avoid it, but in case there is cancer and model predicted that the user does not have cancer in which the user will be unaware of cancer and can lead to several diseases in future and life risk with time will be increasing, therefore the predictive model should be avoiding the false negative. The limitation of the manual analysis was seen closely during the Pandemic Covid - 19 when the disease was spreading on a large scale where the major restrictions were regarding availability of the doctors to treat the disease due to which a widespread outbreak was seen all over the world \cite{madubueze2020controlling}.

\textbf{}

 The X-ray images of the respiratory organ are one of the major ways of understanding the illness and providing valuable insights regarding it, therefore, understanding the image and manual interpretation of the disease was used, but the problem with the given method is the X-ray machine is not available for every user and after the X-ray machine generate the image that image needs to be manually analysis by the doctor which led to long queue of patient waiting for the doctor consult to make the decision. But with the predictive model and the application of advanced computational models and techniques, a predictive model based on a machine learning algorithm can be developed as discussed in the above section which offers a quite promising output for curing the disease\cite{rawat2017deep}. The main aim behind the selected research is to search for different deep learning techniques based on Convolutional Neural Networks (CNN), to predict and classify different lung diseases that are  Bacterial Pneumonia, Corona Virus Disease, Normal, Tuberculosis, and Viral Pneumonia, the limitation of the chosen disease is based on the availability of data where the data for all the above-mentioned disease are present in open web for research and learning purpose. 

\textbf{}

Another important aspect of this research is understanding the time and data required to build the model and understanding the training time which is related to the availability of computation resources as the model is based on CNN which requires high computation power along with a graphical processing unit which help in training the model is optimize manner and usage of predictive model is possible because in the development of machine learning algorithm and computation resource as they both are important to get accurate and effective result .. By employing a deep learning model trained on a diverse dataset of lung X-rays\cite{rawat2017deep}. Different evaluation metrics related to the classification algorithm are accuracy score, and f1 score which are useful to understand how well the model is predicting regarding already trained data and the world data and help the developer understand the palace where the model needs to be re configured or tuned parameter to achieve much better result and to achieve maximum score the procure will be to combine prior knowledge regarding the model and data and applying tunning and combination of different neural network in sequential manner which can help in understanding the application of machine learning technique in health care sector\cite{powers2020evaluation}.

\textbf{}

In the past different studies have been conducted to understand if the computation power can be used in the healthcare sector to perform a day-to-day task in an efficient manner which turned out to be very effective, Brown and Jones (2019) conducted a comprehensive review in which the main highlighting was performed in determining the effect of air pollution in day to day life of any user where user data from the different location along with population level is collected where a relation between the air quality and the respiratory disease was seen in which was important to understand because the air is the most common medium through which the disease spread among different patitent\cite{sacks2011particulate}. 


% PLEAE CHANGE THE TITLE of this section
\section{Existing System} 
% Note the use of \cite{} and \citep{}
The existing system for lung disease prediction integrates various components, starting with the collection and preprocessing of medical data obtained from diverse sources such as medical centers and research institutions. This data encompasses patient demographics, medical history, symptoms, and diagnostic tests, including lung imaging studies like X-rays or CT scans. Following data preprocessing to handle missing values and standardize features, relevant characteristics representing lung health and disease are extracted. Machine learning algorithms play a pivotal role in lung disease prediction, including Logistic Regression, Random Forest, Support Vector Machines (SVM), Deep Learning (particularly Convolutional Neural Networks or CNNs), and Gradient Boosting Machines (GBM) like XGBoost or LightGBM. These algorithms are trained on preprocessed data using techniques such as cross-validation to optimize performance and prevent overfitting. Evaluation metrics like accuracy, precision, recall, F1 score, and area under the ROC curve assess model performance. Clinical validation is crucial, involving collaboration with medical experts to validate models in real-world settings. Once validated, these models can be integrated into clinical decision support systems or electronic health record (EHR) systems to aid healthcare providers in diagnosing and managing lung diseases. Ongoing refinement and updates to these algorithms aim to enhance predictive accuracy and clinical utility in lung disease prediction.

% PLEAE CHANGE THE TITLE of this section
\section{Proposed System}
There are multiple crucial steps in the proposed approach for using deep learning techniques in medical picture analysis, especially for the identification of respiratory diseases. First, a large dataset of CT or X-ray pictures of the chest is gathered and produced. Labels for respiratory illnesses such as COVID-19, pneumonia, and tuberculosis are added. Then, suitable deep learning architectures for the task are chosen, such as VGG, ResNet, or MobileNet. These structures are optimized for respiratory disease diagnosis and are well-known for their efficacy in medical picture analysis. After that, a training pipeline is created to preprocess the data, doing things like augmentation, normalization, and scaling. The model may be successfully trained and assessed thanks to the dataset's division into training, validation, and testing sets. In order to increase efficiency during model training, transfer learning techniques are applied, and model parameters are adjusted utilizing optimization algorithms such as Adam. The model's performance is evaluated using measures like accuracy, precision, recall, F1 score, and confusion matrix analysis, with an emphasis on reducing false negatives to guarantee accurate disease identification. Python programming and deep learning frameworks like TensorFlow, Keras, or PyTorch are used to implement the suggested system. 