\chapter{Literature Review}
\label{ch:lit_rev} %Label of the chapter lit rev. The key ``ch:lit_rev'' can be used with command \ref{ch:lit_rev} to refer this Chapter.

In recent years, the growth in different types of infectious diseases has led to the urgent need for any analysis method to understand and cure the disease where the most effective analysis can be performed using predictive models as manual analysis has many drawbacks such as its time-consuming and is hard to analyze the complexity of the illness when it comes to disease cause directly into the respiratory system \cite{haux2019research}. The major agent for illness in the respiratory system is the pollution present in the air and the unhygienic that lead and since some illnesses can be even passed using genetic or body touch therefore the effect count of such diseases grows at a rate of exponential \cite{madubueze2020controlling}. In the past decade, the traditional method has been used to understand the root cause of any given respiratory disease, and based on analysis proper cure is found but with the evolving nature of the illness over time the limitation can be seen where the analysis can take a longer time and same goes for cure in which the best cure regarding the illness need to figure out along with which one important factor to be noted is the predictive model should avoid False Negative for example if there is a minor chance that the user has cancer the model should predict that there is cancer along with probability as it would help in taking proper medication to avoid it, but in case there is cancer and model predicted that the user does not have cancer in which the user will be unaware of cancer and can lead to several diseases in future and life risk with time will be increasing, therefore the predictive model should be avoiding the false negative. The limitation of the manual analysis was seen closely during the Pandemic Covid - 19 when the disease was spreading on a large scale where the major restrictions were regarding availability of the doctors to treat the disease due to which a widespread outbreak was seen all over the world \cite{madubueze2020controlling}.

\textbf{}

 The X-ray images of the respiratory organ are one of the major ways of understanding the illness and providing valuable insights regarding it, therefore, understanding the image and manual interpretation of the disease was used, but the problem with the given method is the X-ray machine is not available for every user and after the X-ray machine generate the image that image needs to be manually analysis by the doctor which led to long queue of patient waiting for the doctor consult to make the decision. But with the predictive model and the application of advanced computational models and techniques, a predictive model based on a machine learning algorithm can be developed as discussed in the above section which offers a quite promising output for curing the disease\cite{rawat2017deep}. The main aim behind the selected research is to search for different deep learning techniques based on Convolutional Neural Networks (CNN), to predict and classify different lung diseases that are  Bacterial Pneumonia, Corona Virus Disease, Normal, Tuberculosis, and Viral Pneumonia, the limitation of the chosen disease is based on the availability of data where the data for all the above-mentioned disease are present in open web for research and learning purpose. 

\textbf{}

Another important aspect of this research is understanding the time and data required to build the model and understanding the training time which is related to the availability of computation resources as the model is based on CNN which requires high computation power along with a graphical processing unit which help in training the model is optimize manner and usage of predictive model is possible because in the development of machine learning algorithm and computation resource as they both are important to get accurate and effective result .. By employing a deep learning model trained on a diverse dataset of lung X-rays\cite{rawat2017deep}. Different evaluation metrics related to the classification algorithm are accuracy score, and f1 score which are useful to understand how well the model is predicting regarding already trained data and the world data and help the developer understand the palace where the model needs to be re configured or tuned parameter to achieve much better result and to achieve maximum score the procure will be to combine prior knowledge regarding the model and data and applying tunning and combination of different neural network in sequential manner which can help in understanding the application of machine learning technique in health care sector\cite{powers2020evaluation}.

\textbf{}

In the past different studies have been conducted to understand if the computation power can be used in the healthcare sector to perform a day-to-day task in an efficient manner which turned out to be very effective, Brown and Jones (2019) conducted a comprehensive review in which the main highlighting was performed in determining the effect of air pollution in day to day life of any user where user data from the different location along with population level is collected where a relation between the air quality and the respiratory disease was seen in which was important to understand because the air is the most common medium through which the disease spread among different patitent\cite{sacks2011particulate}. 


% PLEAE CHANGE THE TITLE of this section
\section{Example of in-text citation of references in \LaTeX} 
% Note the use of \cite{} and \citep{}
The references in a report relate your content with the relevant sources, papers, and the works of others. To include references in a report, we \textit{cite} them in the texts. In MS-Word, EndNote, or MS-Word references, or plain text as a list can be used. Similarly, in \LaTeX, you can use the ``thebibliography'' environment, which is similar to the plain text as a list arrangement like the MS word. However, In \LaTeX, the most convenient way is to use the BibTex, which takes the references in a particular format [see references.bib file of this template] and lists them in style [APA, Harvard, etc.] as we want with the help of proper packages.    

These are the examples of how to \textit{cite} external sources, seminal works, and research papers. In \LaTeX, if you use ``\textbf{BibTex}'' you do not have to worry much since the proper use of a bibliographystyle package like ``agsm for the Harvard style'' and little rectification of the content in a BiBText source file [In this template, BibTex are stored in the ``references.bib'' file], we can conveniently generate  a reference style. 

Take a note of the commands \textbackslash cite\{\} and \textbackslash citep\{\}. The command \textbackslash cite\{\} will write like ``Author et al. (2019)'' style for Harvard, APA and Chicago style. The command \textbackslash citep\{\} will write like ``(Author et al., 2019).'' Depending on how you construct a sentence, you need to use them smartly. Check the examples of \textbf{in-text citation} of sources listed here [This template recommends the \textbf{Harvard style} of referencing.]:
\begin{itemize}
    \item \cite{lamport1994latex} has written a comprehensive guide on writing in \LaTeX ~[Example of \textbackslash cite\{\} ].
    \item If \LaTeX~is used efficiently and effectively, it helps in writing a very high-quality project report~\citep{lamport1994latex} ~[Example of \textbackslash citep\{\} ].   
    \item A detailed APA, Harvard, and Chicago referencing style guide are available in~\citep{uor_refernce_style}.
\end{itemize}

\noindent 
Example of a numbered list:
\begin{enumerate}
    \item \cite{lamport1994latex} has written a comprehensive guide on writing in \LaTeX.
    \item If \LaTeX is used efficiently and effectively, it helps in writing a very high-quality project report~\citep{lamport1994latex}.   
\end{enumerate}

% PLEAE CHANGE THE TITLE of this section
\section{Example of ``risk'' of unintentional plagiarism}
Using other sources, ideas, and material always bring with it a risk of unintentional plagiarism. 

\noindent
\textbf{\color{red}MUST}: do read the university guidelines on the definition of plagiarism as well as the guidelines on how to avoid plagiarism~\citep{uor_plagiarism}.




% A possible section of you chapter
\section{Critique of the review} % Use this section title or choose a betterone
Describe your main findings and evaluation of the literature. ~\\

% Pleae use this section
\section{Summary} 
Write a summary of this chapter~\\
