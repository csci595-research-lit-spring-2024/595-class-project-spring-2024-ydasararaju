\chapter{Introduction}
\label{ch:into} % This how you label a chapter and the key (e.g., ch:into) will be used to refer this chapter ``Introduction'' later in the report. 
% the key ``ch:into'' can be used with command \ref{ch:intor} to refere this Chapter.

It is become harder and harder to understand disease patterns, particularly those associated with respiratory disease \cite{Zimlich}. The dynamic character of these patterns, which change over time, is the source of this issue. As such, it becomes difficult to understand these patterns precisely, which results in insufficient preventative measures and therapies, which in turn encourages the spread of these diseases. Different pulmonary X-ray pictures are frequently used in the investigation of these conditions. Rising pollution levels and deteriorating public health conditions—which are predicted to get worse over time—are the main causes of the rise in these diseases. Massive outbreaks could eventually result from these conditions; one example of this is the Covid-19 virus \cite{WHO}, which mostly attacks the lungs and spreads quickly among people in close proximity. Pandemic scenarios make manual analysis more challenging. But still Predictive modeling is one workable substitute. When compared to manual validation, these models allow for faster and more efficient analysis across several sites.



\section{Background}
\label{sec:into_back}

In this research paper, complex predictive modeling is performed which will be based on a 
machine learning algorithm where the chosen algorithm will be based upon deep learning 
technique, and the input dataset \cite{Kaggle} of the model is based upon different lung X-rays 
which are collected over different periods and are annotated with four different respiratory 
that are Bacterial Pneumonia, Corona Virus Disease, Tuberculosis, Viral 
Pneumonia. The all above-mentioned diseases can be easily spread and can cause a massive 
pandemic in the future, and since the current research question is if the predicting model 
helps understand the pattern of the disease therefore four diseases are sufficient to analyze 
and the performance of the given model will be governed using different classification metric  
\cite{sultana2021using} available. The research will consist of a deep analysis of previous research 
conducted and the use of Python language to train and analyze the predictive model.

%%%%%%%%%%%%%%%%%%%%%%%%%%%%%%%%%%%%%%%%%%%%%%%%%%%%%%%%%%%%%%%%%%%%%%%%%%%%%%%%%%%
\section{Problem statement}
\label{sec:intro_prob_art}
The study uses deep learning algorithms on lung X-ray pictures to accurately understand respiratory illness patterns. Conventional approaches are frequently imprecise, resulting in unsatisfactory outcomes. The project's goal is to create a predictive model that can reliably forecast illness patterns by utilizing cutting-edge algorithms. This will enhance our capacity to manage and mitigate respiratory infections, particularly in light of possible pandemics.

%%%%%%%%%%%%%%%%%%%%%%%%%%%%%%%%%%%%%%%%%%%%%%%%%%%%%%%%%%%%%%%%%%%%%%%%%%%%%%%%%%%
\section{Aims and objectives}
\label{sec:intro_aims_obj}


\textbf{Aims:} The aim is to create and assess a deep learning-based predictive modeling method for identifying and analyzing disease patterns in respiratory disorders using lung X-ray images labeled with Viral Pneumonia, Bacterial Pneumonia and Corona Virus Disease.

\textbf{Objectives:} The objectives of this research is to take a high-quality dataset of annotated lung X-ray images, choose and use deep learning algorithms for predictive modeling, train and optimize the model for precise disease pattern analysis, assess performance through a variety of metrics and comparative analyses, investigate approaches for interpreting model predictions, validate the model's generalize across a range of datasets and populations, and offer suggestions for further study and advancements in predictive modeling methods.



%%%%%%%%%%%%%%%%%%%%%%%%%%%%%%%%%%%%%%%%%%%%%%%%%%%%%%%%%%%%%%%%%%%%%%%%%%%%%%%%%%%
\section{Solution approach}
\label{sec:intro_sol} % label of Org section
In order to predict the lung-related disease based on the provided data set, a predictive model is built that can accurately determine and identify the category of the disease which further helps the health care professionals in early detection and prevention of the disease. First, the collection of data set is done which contains four varieties of lung conditions. Convolutional Neural Network (CNN) is used for the image data sets which is present in the Tensor Flow library. The data set goes through a number of steps like data analysis, followed by data processing in which the data is divided into training, testing, and validation. This distribution helps the model to make a uniform and precise prediction after the model is trained on the training data set given as input. After processing the data, augmentation of the image is done in which the dataset is modified to increase the variability of the data like shifting the height, zooming the image, and increasing the quality of the image which as a result increases the amount as well as the quality of the image. Transfer learning is applied which is a technique of machine learning in which the knowledge gained while pre-training a model is applied to a new model to boost the performance of the new model. The next step is to train the model which will be used to predict the disease. The models used for this predictive modelling are VGG 19, ResNet, and MobileNet. Each of these models is very reliable and efficiently performs on the training data to give the desired outcome. Further, optimizers are used to increase the performance and efficiency of the trained models like Adam optimizer. Other evaluation metrics are used to evaluate the model’s performance like the confusion matrix, classification report, accuracy, and precision recall.

\subsection{A subsection 1}
\label{sec:intro_some_sub1}
You may or may not need subsections here. Depending on your project's needs, add two or more subsection(s). A section takes at least two subsections. 

\subsection{A subsection 2}
\label{sec:intro_some_sub2}
Depending on your project's needs, add more section(s) and subsection(s).

\subsubsection{A subsection 1 of a subsection}
\label{sec:intro_some_subsub1}
The command \textbackslash subsubsection\{\} creates a paragraph heading in \LaTeX.

\subsubsection{A subsection 2 of a subsection}
\label{sec:intro_some_subsub2}
Write your text here...

%%%%%%%%%%%%%%%%%%%%%%%%%%%%%%%%%%%%%%%%%%%%%%%%%%%%%%%%%%%%%%%%%%%%%%%%%%%%%%%%%%%
\section{Summary of contributions and achievements} %  use this section 
\label{sec:intro_sum_results} % label of summary of results
Describe clearly what you have done/created/achieved and what the major results and their implications are. 


%%%%%%%%%%%%%%%%%%%%%%%%%%%%%%%%%%%%%%%%%%%%%%%%%%%%%%%%%%%%%%%%%%%%%%%%%%%%%%%%%%%


\textbf{Example: how to refer a chapter, section, subsection}. This report is organised into seven chapters. Chapter~\ref{ch:lit_rev} details the literature review of this project. In Section~\ref{ch:method}...  % and so on.

\textbf{Note:}  Take care of the word like ``Chapter,'' ``Section,'' ``Figure'' etc. before the \LaTeX command \textbackslash ref\{\}. Otherwise, a  sentence will be confusing. For example, In \ref{ch:lit_rev} literature review is described. In this sentence, the word ``Chapter'' is missing. Therefore, a reader would not know whether 2 is for a Chapter or a Section or a Figure.