\chapter{Introduction}
\label{ch:into} % This how you label a chapter and the key (e.g., ch:into) will be used to refer this chapter ``Introduction'' later in the report. 
% the key ``ch:into'' can be used with command \ref{ch:intor} to refere this Chapter.

The manual study of illness patterns especially related to respiratory disease\cite{Zimlich} is becoming tough to understand as their pattern changes with time and due to this the proper precautions and medicine are not taken which leads to the growth of such diseases and the analysis of all such diseases are performed using different X-ray images of the lungs and the main reason behind the growth of such disease is the increasing pollution and bad public health which will continue to grow with time and will finally turn into a massive outbreak of the disease one of such example is the Covid-19\cite{WHO} virus  which was spread in the lungs and people got infected near by, and in such pandemic situation the manual analysis is quite hard, but performing the same analysis using different predicting modeling is quite easy as it can be installed at multiple locations and can perform more quickie and effectively as compared to manual validation. 

In this research paper, complex predictive modeling is performed which will be based on a machine learning algorithm where the chosen algorithm will be based upon deep learning technique, and the input dataset \cite{Kaggle} of the model is based upon different lung X-rays which are collected over different periods and are annotated with four different respiratory that are “Bacterial Pneumonia”, “Corona Virus Disease”, “Normal”, “Tuberculosis”, “Viral Pneumonia”. The all above-mentioned diseases can be easily spread and can cause a massive pandemic in the future, and since the current research question is if the predicting model helps understand the pattern of the disease therefore four diseases are sufficient to analyze and the performance of the given model will be governed using different classification metric \cite{JP} available. The research will consist of a deep analysis of previous research conducted and the use of Python language to train and analyze the predictive model.


