\chapter{Introduction}
\label{ch:into} % This how you label a chapter and the key (e.g., ch:into) will be used to refer this chapter ``Introduction'' later in the report. 
% the key ``ch:into'' can be used with command \ref{ch:intor} to refere this Chapter.

The manual study of illness patterns especially related to respiratory disease\cite{Zimlich} is becoming tough to understand as their pattern changes with time and due to this the proper precautions and medicine are not taken which leads to the growth of such diseases and the analysis of all such diseases are performed using different X-ray images of the lungs and the main reason behind the growth of such disease is the increasing pollution and bad public health which will continue to grow with time and will finally turn into a massive outbreak of the disease one of such example is the Covid-19\cite{WHO} virus  which was spread in the lungs and people got infected near by, and in such pandemic situation the manual analysis is quite hard, but performing the same analysis using different predicting modeling is quite easy as it can be installed at multiple locations and can perform more quickie and effectively as compared to manual validation. 

In this research paper, complex predictive modeling is performed which will be based on a machine learning algorithm where the chosen algorithm will be based upon deep learning technique, and the input dataset \cite{Kaggle} of the model is based upon different lung X-rays which are collected over different periods and are annotated with four different respiratory that are “Bacterial Pneumonia”, “Corona Virus Disease”, “Normal”, “Tuberculosis”, “Viral Pneumonia”. The all above-mentioned diseases can be easily spread and can cause a massive pandemic in the future, and since the current research question is if the predicting model helps understand the pattern of the disease therefore four diseases are sufficient to analyze and the performance of the given model will be governed using different classification metric \cite{JP} available. The research will consist of a deep analysis of previous research conducted and the use of Python language to train and analyze the predictive model.
\section{Background}
\label{sec:into_back}
Describe to a reader the context of your project. That is, what is your project and what its motivation. Briefly explain the major theories, applications, and/or products/systems/algorithms whichever is relevant to your project.

\textbf{Cautions:} Do not say you choose this project because of your interest, or your supervisor proposed/suggested this project, or you were assigned this project as your final year project. This all may be true, but it is not meant to be written here.

%%%%%%%%%%%%%%%%%%%%%%%%%%%%%%%%%%%%%%%%%%%%%%%%%%%%%%%%%%%%%%%%%%%%%%%%%%%%%%%%%%%
\section{Problem statement}
\label{sec:intro_prob_art}
How can the implementation of machine learning algorithms, utilizing historical health data, achieve a minimum accuracy of 90percent in predicting the onset of infectious disease outbreaks within a specific population by the end of the next academic year, and what impact does the timely intervention resulting from these predictions have on mitigating the spread and severity of the outbreaks?

%%%%%%%%%%%%%%%%%%%%%%%%%%%%%%%%%%%%%%%%%%%%%%%%%%%%%%%%%%%%%%%%%%%%%%%%%%%%%%%%%%%
\section{Aims and objectives}
\label{sec:intro_aims_obj}
Describe the ``aims and objectives'' of your project. 

\textbf{Aims:} The aims tell a read what you want/hope to achieve at the end of the project. The  aims define your intent/purpose in general terms.  

\textbf{Objectives:} The objectives are a set of tasks you would perform in order to achieve the defined aims. The objective statements have to be specific and measurable through the results and outcome of the project.



%%%%%%%%%%%%%%%%%%%%%%%%%%%%%%%%%%%%%%%%%%%%%%%%%%%%%%%%%%%%%%%%%%%%%%%%%%%%%%%%%%%
\section{Solution approach}
\label{sec:intro_sol} % label of Org section
Briefly describe the solution approach and the methodology applied in solving the set aims and objectives.

Depending on the project, you may like to alter the ``heading'' of this section. Check with you supervisor. Also, check what subsection or any other section that can be added in or removed from this template.

\subsection{A subsection 1}
\label{sec:intro_some_sub1}
You may or may not need subsections here. Depending on your project's needs, add two or more subsection(s). A section takes at least two subsections. 

\subsection{A subsection 2}
\label{sec:intro_some_sub2}
Depending on your project's needs, add more section(s) and subsection(s).

\subsubsection{A subsection 1 of a subsection}
\label{sec:intro_some_subsub1}
The command \textbackslash subsubsection\{\} creates a paragraph heading in \LaTeX.

\subsubsection{A subsection 2 of a subsection}
\label{sec:intro_some_subsub2}
Write your text here...

%%%%%%%%%%%%%%%%%%%%%%%%%%%%%%%%%%%%%%%%%%%%%%%%%%%%%%%%%%%%%%%%%%%%%%%%%%%%%%%%%%%
\section{Summary of contributions and achievements} %  use this section 
\label{sec:intro_sum_results} % label of summary of results
Describe clearly what you have done/created/achieved and what the major results and their implications are. 


%%%%%%%%%%%%%%%%%%%%%%%%%%%%%%%%%%%%%%%%%%%%%%%%%%%%%%%%%%%%%%%%%%%%%%%%%%%%%%%%%%%
\section{Organization of the report} %  use this section
\label{sec:intro_org} % label of Org section
Describe the outline of the rest of the report here. Let the reader know what to expect ahead in the report. Describe how you have organized your report. 

\textbf{Example: how to refer a chapter, section, subsection}. This report is organised into seven chapters. Chapter~\ref{ch:lit_rev} details the literature review of this project. In Section~\ref{ch:method}...  % and so on.

\textbf{Note:}  Take care of the word like ``Chapter,'' ``Section,'' ``Figure'' etc. before the \LaTeX command \textbackslash ref\{\}. Otherwise, a  sentence will be confusing. For example, In \ref{ch:lit_rev} literature review is described. In this sentence, the word ``Chapter'' is missing. Therefore, a reader would not know whether 2 is for a Chapter or a Section or a Figure.