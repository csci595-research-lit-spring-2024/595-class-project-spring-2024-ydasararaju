\chapter{Discussion and Analysis}

\section{Discussion}
Respiratory-related infections and diseases are a significant obstacle faced worldwide by various health sectors and the general public, especially in old age or places with dense air pollution which leads to deteriorated health conditions. Analyzing such factors manually which cause lung illness is very complicated, laborious, and error-prone, especially in times of global pandemic outbreaks like COVID-19. To better understand four common respiratory diseases—tuberculosis, coronavirus illness, bacterial pneumonia, and normal conditions—this research explores a deep learning-based predictive modeling approach that uses different machine learning and deep learning algorithms to analyze the patient’s medical records for early identification of any traces of lung infection and also preventive measures. The objective of this research is to build an effective healthcare prediction model using different deep learning networks and models that can accurately identify and predict whether the patient is suffering from any classes of lung disease mentioned in the dataset

\section{Summary}
The project explores various processes including predictive modeling techniques like data reading, data analysis where the input dataset is cleaned and modified from raw data to useful data, data processing where the input data is processed for extracting useful information, data visualization where various factors are presented in the form of graphs which is easy to understand and analyze followed by data augmentation where the quality of the input image is enhanced utilizing various techniques like cropping, resizing, upscaling, zooming, etc. All of these processes can be done by using Python libraries like Pandas, Numpy, TensorFlow, and Matplotlib which are built for such tasks. The final step is to deploy various models and train them by providing them with a set of training and testing data. For this research, models like VGG 19, MobileNet, and ResNet are used which are combined with Adam Optimizer and transfer learning which further increases the model’s performance, efficiency, and reliable predictions for any early signs of lung disease. Training these Recurrent Neural Network models is very complex as there are many challenges like underfitting and overfitting, normalization, etc. At last, these models are evaluated based on various performance metrics like classification report, confusion matrix, accuracy score, and F1 score. This research advances the field of predictive analytics in healthcare by combining cutting-edge machine learning approaches with thorough evaluation measures. In the end, it improves patient outcomes globally by assisting healthcare systems in better managing and preventing outbreaks of lung illness.
